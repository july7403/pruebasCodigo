\section{Hipótesis}

Durante el desarrollo del TP se establecieron ciertas hipótesis con el objetivo de delimitar el alcance del sistema y facilitar su implementación. 

Se asumió que:
\begin{itemize}
    \item Las bases de conocimiento están bien formadas y que las cláusulas se expresan de manera sintacticamente correcta.
    \item Los hechos y reglas ingresados no contienen recursión infinita ni construcciones que impidan la finalización del proceso de inferencia.
    \item La unificación implementada es suficiente para resolver los objetivos propuestos, sin necesidad de extenderla a operadores aritméticos o estructuras de datos complejas.
    \item Las variables son locales a cada consulta y que no se requiere un manejo de entornos anidados o contextos múltiples.
    \item Mundo cerrado: toda consulta que no pueda ser unificada o resuelta se considera falsa.
    \item Por ultimo, se consideró que la interacción con el usuario, ya sea por consola o interfaz, no afecta el funcionamiento lógico del motor y se limita a la carga y visualización de resultados.
    
\end{itemize}