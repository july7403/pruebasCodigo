\section{Conclusiones}

El desarrollo de este TP nos permitió entender de manera profunda el funcionamiento interno de un motor de inferencia basado en lógica de primer orden, similar al utilizado por Prolog. A través de la implementación en \textbf{Scala 3}, se exploraron conceptos fundamentales, tales como la unificación, la resolución de metas y el backtracking.

Durante la construcción del módulo logico en \textbf{engine.scala}, se logró implementar un sistema de resolución funcional que integra los componentes principales: el parser encargado del análisis sintactico de la base de conocimiento, el módulo de unificación que maneja las sustituciones entre términos, y el motor de inferencia que aplica recursivamente las reglas de resolución para encontrar soluciones válidas a las consultas.

La integración con \textbf{main.scala} permitió tener una interfaz de ejecución completa, capaz de leer una base de conocimiento, procesar consultas y mostrar resultados mediante sustituciones válidas. De esta forma, el motor desarrollado cumple con los principios del razonamiento lógico automatizado y demuestra un comportamiento equivalente al de un intérprete Prolog básico.

En términos de aprendizaje, este trabajo ayudo a la comprensión del paradigma lógico y su relación con los mecanismos de búsqueda y deducción, así como la importancia del manejo de variables, la estandarización de nombres y en general las buenas practicas de programacion. Además, el uso de Scala aportó una visión funcional del problema, facilitando la implementación de estructuras inmutables.

En conclusión, el trabajo permitió no solo construir un sistema operativo y coherente de inferencia lógica, sino también afianzar habilidades de diseño, abstracción y razonamiento formal en el contexto de la programación funcional.