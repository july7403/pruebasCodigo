\section{Compilación y ejecución de Tests:}

Previo a la ejecución del programa debe verificarse tener instaladas las herramientas pertinentes:
\begin{itemize}
	\item sbt (Scala Build Tool)
	\item java 8 o superior
	\item Sistema operativo Unix/Linux/macOS
\end{itemize}
	
\subsection{Ejecución de múltiples tests:}
La ejecución de todos los tests puede realizarse fácilmente con $\texttt{./test\_runner.sh}$. Este script procede a compilar el código para luego ejecutar secuencialmente uno a uno los diferentes programas de ejemplos, tomando el programa con terminación $\texttt{.pl}$ y procesandolo, luego toma las consultas del archivo $\texttt{.txt}$ de mismo nombre para realizar las consultas indicadas, tras esto va a buscar el resultado esperado en el archivo de mismo nombre con terminación $\texttt{.res}$ y compara lo devuelto por el programa con el resultado esperado e indica si es correcto o no (en este caso indicando que fue lo recibido y lo esperado). 

\subsection{Ejecución de un programa:}
Para este caso desde la raiz del proyecto puede ejecutarse por la terminal mediante $sbt$ con el siguiente comando:

\verb|sbt "run <path/de/entrada.pl> <path/de/input.txt>"|

Opcionalmente se puede ejecutar con el input $"-"$ para que el programa tome las consultas por $STDIN$. Este proceso termina cuando el usuario ingresa la consulta $EXIT$.


\pagebreak

